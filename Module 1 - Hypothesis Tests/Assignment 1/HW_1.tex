\documentclass[]{article}
\usepackage{lmodern}
\usepackage{amssymb,amsmath}
\usepackage{ifxetex,ifluatex}
\usepackage{fixltx2e} % provides \textsubscript
\ifnum 0\ifxetex 1\fi\ifluatex 1\fi=0 % if pdftex
  \usepackage[T1]{fontenc}
  \usepackage[utf8]{inputenc}
\else % if luatex or xelatex
  \ifxetex
    \usepackage{mathspec}
  \else
    \usepackage{fontspec}
  \fi
  \defaultfontfeatures{Ligatures=TeX,Scale=MatchLowercase}
\fi
% use upquote if available, for straight quotes in verbatim environments
\IfFileExists{upquote.sty}{\usepackage{upquote}}{}
% use microtype if available
\IfFileExists{microtype.sty}{%
\usepackage{microtype}
\UseMicrotypeSet[protrusion]{basicmath} % disable protrusion for tt fonts
}{}
\usepackage[margin=1in]{geometry}
\usepackage{hyperref}
\hypersetup{unicode=true,
            pdftitle={Homework 1},
            pdfauthor={Scott Girten},
            pdfborder={0 0 0},
            breaklinks=true}
\urlstyle{same}  % don't use monospace font for urls
\usepackage{longtable,booktabs}
\usepackage{graphicx,grffile}
\makeatletter
\def\maxwidth{\ifdim\Gin@nat@width>\linewidth\linewidth\else\Gin@nat@width\fi}
\def\maxheight{\ifdim\Gin@nat@height>\textheight\textheight\else\Gin@nat@height\fi}
\makeatother
% Scale images if necessary, so that they will not overflow the page
% margins by default, and it is still possible to overwrite the defaults
% using explicit options in \includegraphics[width, height, ...]{}
\setkeys{Gin}{width=\maxwidth,height=\maxheight,keepaspectratio}
\IfFileExists{parskip.sty}{%
\usepackage{parskip}
}{% else
\setlength{\parindent}{0pt}
\setlength{\parskip}{6pt plus 2pt minus 1pt}
}
\setlength{\emergencystretch}{3em}  % prevent overfull lines
\providecommand{\tightlist}{%
  \setlength{\itemsep}{0pt}\setlength{\parskip}{0pt}}
\setcounter{secnumdepth}{0}
% Redefines (sub)paragraphs to behave more like sections
\ifx\paragraph\undefined\else
\let\oldparagraph\paragraph
\renewcommand{\paragraph}[1]{\oldparagraph{#1}\mbox{}}
\fi
\ifx\subparagraph\undefined\else
\let\oldsubparagraph\subparagraph
\renewcommand{\subparagraph}[1]{\oldsubparagraph{#1}\mbox{}}
\fi

%%% Use protect on footnotes to avoid problems with footnotes in titles
\let\rmarkdownfootnote\footnote%
\def\footnote{\protect\rmarkdownfootnote}

%%% Change title format to be more compact
\usepackage{titling}

% Create subtitle command for use in maketitle
\newcommand{\subtitle}[1]{
  \posttitle{
    \begin{center}\large#1\end{center}
    }
}

\setlength{\droptitle}{-2em}

  \title{Homework 1}
    \pretitle{\vspace{\droptitle}\centering\huge}
  \posttitle{\par}
    \author{Scott Girten}
    \preauthor{\centering\large\emph}
  \postauthor{\par}
    \date{}
    \predate{}\postdate{}
  

\begin{document}
\maketitle

All homework will be submitted via R Markdown files.

\subsubsection{Question 1}\label{question-1}

Complete problem 3.5 (page 71) in the text. The two requested hypothesis
tests should follow the format below:

\begin{itemize}
\tightlist
\item
  Define the parameters of interest and state the hypotheses to be
  tested.
\item
  Define the test statistic used to test the claim.
\item
  Simulate the null distribution of the test statistic, including a
  histogram of the distribution with the observed test statistic
  identified.
\item
  Caclulate the p-value of the test.
\item
  Report a conclusion for the test.
\end{itemize}

\begin{center}\rule{0.5\linewidth}{\linethickness}\end{center}

\(\underline{\textbf{Part A}}\)

\begin{quote}
\(\mu_a\) = mean delay time for American Airlines

\(\mu_u\) = mean delay time for United Airlines

\(H_o : \mu_a = \mu_u\)

\(H_a : \mu_a \ne \mu_u\)
\end{quote}

\textbf{Test Statistic:} if the mean delay times for American Airlines
and United Airlines is statistically significant, the test statistic
\(T(X) = \bar{X}_a - \bar{X}_u\) can be used.

Mean of \(T(X)\) = 1.1643652

Standard Deviation of \(T(X)\) = 0.8833419

\textbf{p-value Calculation}

The p-value for \(T(X)\) is 10\^{}\{-4\}

\subsubsection{Question 2}\label{question-2}

Complete problem 3.7 (page 71). The problem asks you to repeat part a of
question 3.5 using 3 different test statistics. Please make sure the
caclulation of the 3 test statistics all occurs within the same FOR
loop.

\subsubsection{Question 3}\label{question-3}

The Department of Tourism in Illinois wants to determine if there is a
difference in the mean cost of a taxi ride for customers paying by cash
and those paying by credit. A sample of 150 taxi rides is selected, and
the total fare and method of payment is recorded. The data for 78 cash
payments and 72 credit payment is below:

Suppose a permutation test was conducted on this data to determine if
there is a difference in the mean taxi fare for the two payment methods.

\begin{itemize}
\item
  If the \emph{exact} null distribution were constructed, how many
  unique resamples exist?
\item
  With a much larger pool of resamples, let's investigate the effect
  number of resamples has on the mean and varaibility of p-values from
  the permutation test. Write code to generate output that will allow
  the table below to be filled in based on the simulation of 1000
  p-values (note to save computing time, I've provided the values for
  20,000 resamples):
\end{itemize}

\begin{longtable}[]{@{}lll@{}}
\toprule
Number of Resamples & Mean of the p-values & Std Dev of the
p-values\tabularnewline
\midrule
\endhead
1000 & &\tabularnewline
5000 & &\tabularnewline
10,000 & &\tabularnewline
15,000 & &\tabularnewline
20,000 & 0.8837033 & 0.002207726\tabularnewline
\bottomrule
\end{longtable}

\begin{itemize}
\tightlist
\item
  Breifly comment on the pattern observed in the means and standard
  deviations of the p-values.
\end{itemize}

\begin{center}\rule{0.5\linewidth}{\linethickness}\end{center}

The data refereneced in the above questions is accessible at this
\href{https://sites.google.com/site/ChiharaHesterberg}{link}. Individual
files can be downloaded from the
\href{https://sites.google.com/site/chiharahesterberg/data2}{Data sets}
link. Or, an R package has been created with all the data from the text
included. The package can be installed with the following code:

The data you'll be using contains flight delay information for United
Airlines and American Airlines departures out of LaGuardia. Information
on the variables contained in the datset is given in Section 1.1 of the
textbook.

The hypothesis tests you'll be conducting will required you to look at
subsets of the dataset. For example, suppose we wanted to create a
vector of delay times for all United Airlines flights. Assuming the data
has been loaded as a Data Frame with the name FlightDelays, the
following code would accomplish this:


\end{document}
